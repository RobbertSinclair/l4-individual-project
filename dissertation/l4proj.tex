% REMEMBER: You must not plagiarise anything in your report. Be extremely careful.

\documentclass{l4proj}

    
%
% put any additional packages here
%

\begin{document}

%==============================================================================
%% METADATA
\title{A Seamful Game Based on GPS Shadows}
\author{Robbert Malcolm Sinclair}
\date{14 October 2022}

\maketitle

%==============================================================================
%% ABSTRACT
\begin{abstract}
    Every abstract follows a similar pattern. Motivate; set aims; describe work; explain results.
    \vskip 0.5em
    ``XYZ is bad. This project investigated ABC to determine if it was better. 
    ABC used XXX and YYY to implement ZZZ. This is particularly interesting as XXX and YYY have
    never been used together. It was found that  
    ABC was 20\% better than XYZ, though it caused rabies in half of subjects.''
\end{abstract}

%==============================================================================

% EDUCATION REUSE CONSENT FORM
% If you consent to your project being shown to future students for educational purposes
% then insert your name and the date below to  sign the education use form that appears in the front of the document. 
% You must explicitly give consent if you wish to do so.
% If you sign, your project may be included in the Hall of Fame if it scores particularly highly.
%
% Please note that you are under no obligation to sign 
% this declaration, but doing so would help future students.
%
\def\consentname {Robbert Malcolm Sinclair} % your full name
\def\consentdate {14 October 2022} % the date you agree
%
\educationalconsent


%==============================================================================
\tableofcontents

%==============================================================================
%% Notes on formatting
%==============================================================================
% The first page, abstract and table of contents are numbered using Roman numerals and are not
% included in the page count. 
%
% From now on pages are numbered
% using Arabic numerals. Therefore, immediately after the first call to \chapter we need the call
% \pagenumbering{arabic} and this should be called once only in the document. 
%
% Do not alter the bibliography style.
%
% The first Chapter should then be on page 1. You are allowed 40 pages for a 40 credit project and 30 pages for a 
% 20 credit report. This includes everything numbered in Arabic numerals (excluding front matter) up
% to but excluding the appendices and bibliography.
%
% You must not alter text size (it is currently 10pt) or alter margins or spacing.
%
%
%==================================================================================================================================
%
% IMPORTANT
% The chapter headings here are **suggestions**. You don't have to follow this model if
% it doesn't fit your project. Every project should have an introduction and conclusion,
% however. 
%
%==================================================================================================================================
\chapter{Introduction}

% reset page numbering. Don't remove this!
\pagenumbering{arabic} 


Why should the reader care about what are you doing and what are you actually doing?
\section{Guidance}

\textbf{Motivate} first, then state the general problem clearly. 

\section{Writing guidance}
\subsection{Who is the reader?}

This is the key question for any writing. Your reader:

\begin{itemize}
    \item
    is a trained computer scientist: \emph{don't explain basics}.
    \item
    has limited time: \emph{keep on topic}.
    \item
    has no idea why anyone would want to do this: \emph{motivate clearly}
    \item
    might not know \emph{anything} about your project in particular:
    \emph{explain your project}.
    \item
    but might know precise details and check them: \emph{be precise and
    strive for accuracy.}
    \item
    doesn't know or care about you: \emph{personal discussions are
    irrelevant}.
\end{itemize}

Remember, you will be marked by your supervisor and one or more members
of staff. You might also have your project read by a prize-awarding
committee or possibly a future employer. Bear that in mind.

\subsection{References and style guides}
There are many style guides on good English writing. You don't need to
read these, but they will improve how you write.

\begin{itemize}
    \item
    \emph{How to write a great research paper} \cite{Pey17} (\textbf{recommended}, even though you aren't writing a research paper)
    \item
    \emph{How to Write with Style} \cite{Von80}. Short and easy to read. Available online.
    \item
    \emph{Style: The Basics of Clarity and Grace} \cite{Wil09} A very popular modern English style guide.
    \item
    \emph{Politics and the English Language} \cite{Orw68}  A famous essay on effective, clear writing in English.
    \item
    \emph{The Elements of Style} \cite{StrWhi07} Outdated, and American, but a classic.
    \item
    \emph{The Sense of Style} \cite{Pin15} Excellent, though quite in-depth.
\end{itemize}

\subsubsection{Citation styles}

\begin{itemize}
\item If you are referring to a reference as a noun, then cite it as: ``\citet{Orw68} discusses the role of language in political thought.''
\item If you are referring implicitly to references, use: ``There are many good books on writing \citep{Orw68, Wil09, Pin15}.''
\end{itemize}

There is a complete guide on good citation practice by Peter Coxhead available here: \url{http://www.cs.bham.ac.uk/~pxc/refs/index.html}. 
If you are unsure about how to cite online sources, please see this guide: \url{https://student.unsw.edu.au/how-do-i-cite-electronic-sources}.

\subsection{Plagiarism warning}

\begin{highlight_title}{WARNING}
    
    If you include material from other sources without full and correct attribution, you are commiting plagiarism. The penalties for plagiarism are severe.
    Quote any included text and cite it correctly. Cite all images, figures, etc. clearly in the caption of the figure.
\end{highlight_title}


%==================================================================================================================================
\chapter{Background}

In this chapter I will go over some of the foundational concepts required to understnad the background of the project

\section{Introduction to the Global Positioning System (GPS)}

The Global Positioning System also known as GPS is one of the worlds most famous Global Navigation Satellite Systems (GNSS). It allows people to determine their
location from anywhere around the world at any time. GPS has three main phases, the space phase, the control phase and the user phase.
In the space phase, the earth is split up into 24 "slots" which should be filled up by an operational satellite \citep{spsStandard}. Therefore
the GPS constellation is made up of at least 24 opterational satellites. In an ideal situation a user should have access to at least 4 of the satellites
at any one time \citep{Rabbany2006}. 

Each GPS satellite consistently sends down radio signals down to the Earth's surface. As of 2020 there are three main frequencies
these are L1 which emits at 1575.42 MHz, L2 which emits radio waves at 1227.6 MHz and L5 with 1176.45 MHz \citep{spsStandard}. This project
will most likely be using L2 and L1 as L5 is used mainly for aviation safety \citep{Xu2016} A GPS receiver will pick up this data and can use the clock signals to calculate the distance from each satellite. Once a 
satellite distance is calculated we can create a radius of equidistant points. We can then add the signals from several other
satellites to find the intersection points. We usually need 3 to 4 satellites to be able to pinpoint the GPS receiver on the map. \citep{Rabbany2006}



%TODO - Add a source and paragraph about how GPS is calculated

\section{What causes Inaccuracies in GPS Systems}
In this project we will be looking at making a game based on GPS shadows which are areas where the GPS signal is not as accurate. In this section I will look
at the various factors that could effect our accuracy.

\subsection{GPS Clock Errors}
% ROUGH DRAFT - Rewrite this with proper sources when you've found them

The first form of GPS error is a clock error. Each satellite sends signals down using an atomic clock which while accurate, can still have a 
slight delay. This delay will be small (a few nanoseconds long) but will accumulate over time. One of the main reasons why this can lead to
a decrease in accuracy is that the receiver may get a time on the clock which is off the actual time on the satellites clock. This can lead to
a slight change of position. This form of error is common with most GPS signals. A typical GPS Receiver can also be a cause for this problem as it 
has a crystal clock which is not as precise than the cesium atomic clocks on the satellites \citep{Rabbany2006, Kleusberg1990}. While this
form of error is a common cause of problem with GPS accuracy, there aren't a lot of game opportunities due to the fact that the player
can't really control when a clock error is to happen.

\subsection{GPS Signal Reception}
GPS signals are in the form of radio waves which can be blocked by obstacles on the way to a GPS sensor. When signals hit a specific obstable
the signals decrese in quality during a process known as attenuation \citep{Indoor2010}. Some examples of such obstacles include
building walls, soil and water. \citep{Kleusberg1990}. Eventually the GNSS receiver will receive more noise
than the original signal which leads to an inaccurate signal. A good signal is received when the receiver has a good signal from at least 4
satellites. Therefore in urban areas with high rise buildings, one or more satellites could be obstructed. If a game was to be made
using this concept then the game would be best played in dense urban areas.

\section{Seamful Design}


\section{Guidance}
\begin{itemize}    
    \item
      Don't give a laundry list of references.
    \item
      Tie everything you say to your problem.
    \item
      Present an argument.
    \item Think critically; weigh up the contribution of the background and put it in context.    
    \item
      \textbf{Don't write a tutorial}; provide background and cite
      references for further information.
\end{itemize}

%==================================================================================================================================
\chapter{Analysis/Requirements}
In this section I go over the main process that I used to come up with a design for the game

\section{Creating an app to get GPS locations}
Since I didn't have much knowledge of mobile app development. I gathered that the best place to start was to make an app that gets a location. 
I started by doing some research into creating an Android app using Kotlin. The main aim of this app was to get me the latitude and longitude coordinates every second and most importantly, give me the accuracy for the GPS locations. 
In order to get this up and running quickly, I tested two main GeoLocation APIs, the FusedLocationProvider API and the LocationManger which can be found in the Android Documentation.

\subsection{FusedLocationProvider API}
The first API I tried was the FusedLocationProvider API as this was the first one that I found. The FusedLocationProvider API combines both GPS and WiFi geolocation
techniques to try and create an optimal location \citep{fused}. This API allowed me to create a basic location app quickly so that I could start finding GPS locations
within my local area. As figure \ref{fig:fusedhist} shows, the location data that I gathered was skewed towards values with an accuracy of less than 50. While a lot of
points were under the 20 meters area. However as you can see, there was a decent amount of values above 50 meters meaning that the API picked up a lot of shadows.
This API would have been a good one to use for my game as I would have had a lot of shadow areas for a player to hide. However the main flaw was that
it was giving me a combination of both Wifi and GPS coordinates. Thies meant that I was more likely to find shadows in areas with low WiFi. This would
explain why I had a lot of shadow areas when I walked around Kelvingrove Park, an area which I thought would have a very accurate signal due to the lack of buildings.

\begin{figure}
    \centering
    \includegraphics[width=0.6\linewidth]{images/fused_histogram.pdf}    

    \caption{Histogram showing the ranges of accuracy (in meters) of location points picked up by the FusedLocationProvider API}

    % use the notation fig:name to cross reference a figure
    \label{fig:fusedhist} 
\end{figure}

\subsection{LocationManger API}
Since the FusedLocationProvider wasn't appropriate for the project, I needed to find an API that would give me location data based purely on GPS. One of the libraries that I found
from the Android Developers Website was the "LocationManger" API. The LocationManager API gives me more control over what kind of Geolocation I was getting on my phone. Therefore
I was able to ensure that my geolocation was based on a full GNSS navigation system. \citep{locationManager} The other main advantage to this library is that with the use of the "requestLocationUpdates" method and
the "LocationListener" interface, I can easily control what my app does when it receives a new location.

While there are some advantages to using the LocationManager API, it does come with some disadvantages as well. For one thing, there are significantly less shadow areas when using
pure GNSS coordinates. The only real areas where there are major shadows are within buildins, under bridges or in enclosed areas. This is because on my phone, it can track up to 
15 satellites when getting my location. In comparison to figure \ref{fig:fusedhist}, figure \ref{fig:mongohist} shows that there is one big peak in the overall accuracy of all the spots.
When walking around with my app I noticed that for most of the time, my accuracy was 3.795 meters of accuracy. However while at first glance this may be a disadvantage the main advantage for creating a
game is that compared to the Fused Location API. It is better overall for an overall game design because the areas where shadows can be found will be more consistent than the ones
found in the Fused Location API. Therefore the game can be used to teach people where GPS can decrease in quality.

\begin{figure}
    \centering
    \includegraphics[width=0.6\linewidth]{images/mongo_db_histogram.png}

    \caption{Histogram showing the ranges of accuracy (in meters) of location points picked up by the LocaitonManager API}

    \label{fig:mongohist}
\end{figure}

\section{Guidance}
Make it clear how you derived the constrained form of your problem via a clear and logical process. 

%==================================================================================================================================
\chapter{Design}
How is this problem to be approached, without reference to specific implementation details? 
\section{Guidance}
Design should cover the abstract design in such a way that someone else might be able to do what you did, but with a different language or library or tool.

\section{Overview}
\label{game-overview}
The game that I will make is a simple tag game where a player can take advantage of GPS shadows in order to hide from
the chaser. The chaser will be able to see the other users locations in real time if the runner is in an area with
high GPS accuracy. Once a chaser tags a runner, the caught runner becomes the chaser and a 30 second Jail Time is
triggered. During this time the catcher cannot see the runner's latest location or is able to catch a runner.

The game will have a time limit with the first iteration being 15 minutes. This is so that the game can be quick and
easy to play during a study break. I also considered the fact that the game would work well in dense urban environments
with lots of buildings to go into like the University of Glasgow campus. Once the time limit is up, the winner will
be calculated as the person who was the runner for the most amount of time in total.

\section{Architecture}
The game that I will create will utilise a Client-Server Architecture. The client will be a smartphone with a GPS or
GNSS sensor. Several clients will be connected using a WebSocket. The smartphone will get the GNSS data including the
latitude, longitude and accuracy and then will process it and send to the server. The location data will also be sent 
to the UI where the user's overall position will be updated. Through the websocket, the phone will send the GPS data
to the server which will save the users location onto a geospatial database. Depending on certain conditions, the data
will be sent to the rest of the clients.

\subsection{Client (Smartphone)}
The smartphone app will be divided into four main components. These are the GPS Listener, the Communications area, the
Player and the User Interface. When the client first connects to the server, it will give the user a player id and a player
type. This information will be given to the player class which stores the id for any further communications with the server.
Then once that is done, the GPS Listener will start to fetch the GPS coordinates every 2 seconds. This data is then sent
to the User Interface so that the map is updated with the user's location. The GPS coordinates are then sent to the Communication
class where they are sent to the server.

\subsection{Server}


\subsection{Client Server Communication}
\label{communication}
Since a lot of the game will have a lot of multiplayer features, it was important that the network side of the game
was robust to ensure that no player would have an unfair advantage when playing the game. Table \ref{tab:webSocketComms}
sets out the different communication types that are required to make this game.

\begin{table}[]
    \caption{The different communication types to the server  }\label{tab:webSocketComms}
    \resizebox{\textwidth}{!}{\begin{tabular}{|l|l|}
    \hline
    \rowcolor[HTML]{9B9B9B} 
    \multicolumn{1}{|c|}{\textbf{Communication Type}} & \multicolumn{1}{c|}{\textbf{Method and Purpose}}                                             \\ \hline
    CONNECT               & Gives the user their id and their initial player type                                                                \\ \hline
    NEW\_TYPE             & This is triggered if either a new user connects to the server or if a user is caught. This will change the users type \\ \hline
                          &                                                                                                                      \\ \hline
    \end{tabular}}
    
    \end{table}

\section{Mechanics}
In this section I go over the various mechanics that I have designed for the game

\subsection{Player Types}
When the player joins the game, they will be assigned to one of two player types. These are the "Chaser" or the "Runner".
The role of the Chaser is to catch a runner as described in section \ref{catching}. When the catcher does this they give the role of chaser to the runner. The Chaser
will be able to see the real time locations of any runner in the game.

The goal of a Runner is to avoid being caught by the Chaser as mentioned before. The Runner will not have a real time
view of the Chaser but will have the ability to "hide" from them by going into areas with a low GPS density. If their
GPS accuracy is low enough, they will not transmit their location to the chaser. Making them invisible to the chaser.
The Runners will also not transmit any location data during the jail period as mentioned in section \ref{game-overview}.

In previous iterations of my design the Runner would have been able to have seen all of the GPS Shadows within a 50
meter radius. The Shadows would be small red dots on the users maps. However after implementing this feature the screen
would freeze up leading to a poor user experience and it gave the Chaser an unfair advantage as they didn't get any
performance issues with just the Runner locations on the screen.

\subsection{Location Tracking and Catching}
\label{catching}
As mentioned before, both the runner and the chaser transmit data to the server. After a set interval
the latest latitude and longitude coordinates will be sent to the server which will update the player's location
in the database. Once this data is collected the chaser is then able to determine if they have caught a runner.

The chaser will have a 5 meter "no-go zone" which will be the way to determine if a player is caught or not. If
the GPS accuracy is high and a player has entered the Chaser's 5 meter radius, then the runner is considered to be 
"caught". My original idea was to calculate the distance on the smartphone however I realised that the main disadvantages
to this was that it was going to be difficult to track who the chaser was. I also thought that since the main architecture
was a Client-Server system that going for as thin a client as possible would be the best approach due to the computational
limitations of a smartphone.
%==================================================================================================================================
\chapter{Implementation}
What did you do to implement this idea, and what technical achievements did you make?
\section{Guidance}
You can't talk about everything. Cover the high level first, then cover important, relevant or impressive details.



\section{General points}

These points apply to the whole dissertation, not just this chapter.



\subsection{Figures}
\emph{Always} refer to figures included, like Figure \ref{fig:relu}, in the body of the text. Include full, explanatory captions and make sure the figures look good on the page.
You may include multiple figures in one float, as in Figure \ref{fig:synthetic}, using \texttt{subcaption}, which is enabled in the template.



% Figures are important. Use them well.
\begin{figure}
    \centering
    \includegraphics[width=0.5\linewidth]{images/relu.pdf}    

    \caption{In figure captions, explain what the reader is looking at: ``A schematic of the rectifying linear unit, where $a$ is the output amplitude,
    $d$ is a configurable dead-zone, and $Z_j$ is the input signal'', as well as why the reader is looking at this: 
    ``It is notable that there is no activation \emph{at all} below 0, which explains our initial results.'' 
    \textbf{Use vector image formats (.pdf) where possible}. Size figures appropriately, and do not make them over-large or too small to read.
    }

    % use the notation fig:name to cross reference a figure
    \label{fig:relu} 
\end{figure}


\begin{figure}
    \centering
    \begin{subfigure}[b]{0.45\textwidth}
        \includegraphics[width=\textwidth]{images/synthetic.png}
        \caption{Synthetic image, black on white.}
        \label{fig:syn1}
    \end{subfigure}
    ~ %add desired spacing between images, e. g. ~, \quad, \qquad, \hfill etc. 
      %(or a blank line to force the subfigure onto a new line)
    \begin{subfigure}[b]{0.45\textwidth}
        \includegraphics[width=\textwidth]{images/synthetic_2.png}
        \caption{Synthetic image, white on black.}
        \label{fig:syn2}
    \end{subfigure}
    ~ %add desired spacing between images, e. g. ~, \quad, \qquad, \hfill etc. 
    %(or a blank line to force the subfigure onto a new line)    
    \caption{Synthetic test images for edge detection algorithms. \subref{fig:syn1} shows various gray levels that require an adaptive algorithm. \subref{fig:syn2}
    shows more challenging edge detection tests that have crossing lines. Fusing these into full segments typically requires algorithms like the Hough transform.
    This is an example of using subfigures, with \texttt{subref}s in the caption.
    }\label{fig:synthetic}
\end{figure}

\clearpage

\subsection{Equations}

Equations should be typeset correctly and precisely. Make sure you get parenthesis sizing correct, and punctuate equations correctly 
(the comma is important and goes \textit{inside} the equation block). Explain any symbols used clearly if not defined earlier. 

For example, we might define:
\begin{equation}
    \hat{f}(\xi) = \frac{1}{2}\left[ \int_{-\infty}^{\infty} f(x) e^{2\pi i x \xi} \right],
\end{equation}    
where $\hat{f}(\xi)$ is the Fourier transform of the time domain signal $f(x)$.

\subsection{Algorithms}
Algorithms can be set using \texttt{algorithm2e}, as in Algorithm \ref{alg:metropolis}.

% NOTE: line ends are denoted by \; in algorithm2e
\begin{algorithm}
    \DontPrintSemicolon
    \KwData{$f_X(x)$, a probability density function returing the density at $x$.\; $\sigma$ a standard deviation specifying the spread of the proposal distribution.\;
    $x_0$, an initial starting condition.}
    \KwResult{$s=[x_1, x_2, \dots, x_n]$, $n$ samples approximately drawn from a distribution with PDF $f_X(x)$.}
    \Begin{
        $s \longleftarrow []$\;
        $p \longleftarrow f_X(x)$\;
        $i \longleftarrow 0$\;
        \While{$i < n$}
        {
            $x^\prime \longleftarrow \mathcal{N}(x, \sigma^2)$\;
            $p^\prime \longleftarrow f_X(x^\prime)$\;
            $a \longleftarrow \frac{p^\prime}{p}$\;
            $r \longleftarrow U(0,1)$\;
            \If{$r<a$}
            {
                $x \longleftarrow x^\prime$\;
                $p \longleftarrow f_X(x)$\;
                $i \longleftarrow i+1$\;
                append $x$ to $s$\;
            }
        }
    }
    
\caption{The Metropolis-Hastings MCMC algorithm for drawing samples from arbitrary probability distributions, 
specialised for normal proposal distributions $q(x^\prime|x) = \mathcal{N}(x, \sigma^2)$. The symmetry of the normal distribution means the acceptance rule takes the simplified form.}\label{alg:metropolis}
\end{algorithm}

\subsection{Tables}

If you need to include tables, like Table \ref{tab:operators}, use a tool like https://www.tablesgenerator.com/ to generate the table as it is
extremely tedious otherwise. 

\begin{table}[]
    \caption{The standard table of operators in Python, along with their functional equivalents from the \texttt{operator} package. Note that table
    captions go above the table, not below. Do not add additional rules/lines to tables. }\label{tab:operators}
    %\tt 
    \rowcolors{2}{}{gray!3}
    \begin{tabular}{@{}lll@{}}
    %\toprule
    \textbf{Operation}    & \textbf{Syntax}                & \textbf{Function}                            \\ %\midrule % optional rule for header
    Addition              & \texttt{a + b}                          & \texttt{add(a, b)}                                    \\
    Concatenation         & \texttt{seq1 + seq2}                    & \texttt{concat(seq1, seq2)}                           \\
    Containment Test      & \texttt{obj in seq}                     & \texttt{contains(seq, obj)}                           \\
    Division              & \texttt{a / b}                          & \texttt{div(a, b) }  \\
    Division              & \texttt{a / b}                          & \texttt{truediv(a, b) } \\
    Division              & \texttt{a // b}                         & \texttt{floordiv(a, b)}                               \\
    Bitwise And           & \texttt{a \& b}                         & \texttt{and\_(a, b)}                                  \\
    Bitwise Exclusive Or  & \texttt{a \textasciicircum b}           & \texttt{xor(a, b)}                                    \\
    Bitwise Inversion     & \texttt{$\sim$a}                        & \texttt{invert(a)}                                    \\
    Bitwise Or            & \texttt{a | b}                          & \texttt{or\_(a, b)}                                   \\
    Exponentiation        & \texttt{a ** b}                         & \texttt{pow(a, b)}                                    \\
    Identity              & \texttt{a is b}                         & \texttt{is\_(a, b)}                                   \\
    Identity              & \texttt{a is not b}                     & \texttt{is\_not(a, b)}                                \\
    Indexed Assignment    & \texttt{obj{[}k{]} = v}                 & \texttt{setitem(obj, k, v)}                           \\
    Indexed Deletion      & \texttt{del obj{[}k{]}}                 & \texttt{delitem(obj, k)}                              \\
    Indexing              & \texttt{obj{[}k{]}}                     & \texttt{getitem(obj, k)}                              \\
    Left Shift            & \texttt{a \textless{}\textless b}       & \texttt{lshift(a, b)}                                 \\
    Modulo                & \texttt{a \% b}                         & \texttt{mod(a, b)}                                    \\
    Multiplication        & \texttt{a * b}                          & \texttt{mul(a, b)}                                    \\
    Negation (Arithmetic) & \texttt{- a}                            & \texttt{neg(a)}                                       \\
    Negation (Logical)    & \texttt{not a}                          & \texttt{not\_(a)}                                     \\
    Positive              & \texttt{+ a}                            & \texttt{pos(a)}                                       \\
    Right Shift           & \texttt{a \textgreater{}\textgreater b} & \texttt{rshift(a, b)}                                 \\
    Sequence Repetition   & \texttt{seq * i}                        & \texttt{repeat(seq, i)}                               \\
    Slice Assignment      & \texttt{seq{[}i:j{]} = values}          & \texttt{setitem(seq, slice(i, j), values)}            \\
    Slice Deletion        & \texttt{del seq{[}i:j{]}}               & \texttt{delitem(seq, slice(i, j))}                    \\
    Slicing               & \texttt{seq{[}i:j{]}}                   & \texttt{getitem(seq, slice(i, j))}                    \\
    String Formatting     & \texttt{s \% obj}                       & \texttt{mod(s, obj)}                                  \\
    Subtraction           & \texttt{a - b}                          & \texttt{sub(a, b)}                                    \\
    Truth Test            & \texttt{obj}                            & \texttt{truth(obj)}                                   \\
    Ordering              & \texttt{a \textless b}                  & \texttt{lt(a, b)}                                     \\
    Ordering              & \texttt{a \textless{}= b}               & \texttt{le(a, b)}                                     \\
    % \bottomrule
    \end{tabular}
    \end{table}
\subsection{Code}

Avoid putting large blocks of code in the report (more than a page in one block, for example). Use syntax highlighting if possible, as in Listing \ref{lst:callahan}.

\begin{lstlisting}[language=python, float, caption={The algorithm for packing the $3\times 3$ outer-totalistic binary CA successor rule into a 
    $16\times 16\times 16\times 16$ 4 bit lookup table, running an equivalent, notionally 16-state $2\times 2$ CA.}, label=lst:callahan]
    def create_callahan_table(rule="b3s23"):
        """Generate the lookup table for the cells."""        
        s_table = np.zeros((16, 16, 16, 16), dtype=np.uint8)
        birth, survive = parse_rule(rule)

        # generate all 16 bit strings
        for iv in range(65536):
            bv = [(iv >> z) & 1 for z in range(16)]
            a, b, c, d, e, f, g, h, i, j, k, l, m, n, o, p = bv

            # compute next state of the inner 2x2
            nw = apply_rule(f, a, b, c, e, g, i, j, k)
            ne = apply_rule(g, b, c, d, f, h, j, k, l)
            sw = apply_rule(j, e, f, g, i, k, m, n, o)
            se = apply_rule(k, f, g, h, j, l, n, o, p)

            # compute the index of this 4x4
            nw_code = a | (b << 1) | (e << 2) | (f << 3)
            ne_code = c | (d << 1) | (g << 2) | (h << 3)
            sw_code = i | (j << 1) | (m << 2) | (n << 3)
            se_code = k | (l << 1) | (o << 2) | (p << 3)

            # compute the state for the 2x2
            next_code = nw | (ne << 1) | (sw << 2) | (se << 3)

            # get the 4x4 index, and write into the table
            s_table[nw_code, ne_code, sw_code, se_code] = next_code

        return s_table

\end{lstlisting}

%==================================================================================================================================
\chapter{Evaluation} 
How good is your solution? How well did you solve the general problem, and what evidence do you have to support that?

\section{Guidance}
\begin{itemize}
    \item
        Ask specific questions that address the general problem.
    \item
        Answer them with precise evidence (graphs, numbers, statistical
        analysis, qualitative analysis).
    \item
        Be fair and be scientific.
    \item
        The key thing is to show that you know how to evaluate your work, not
        that your work is the most amazing product ever.
\end{itemize}

\section{Evidence}
Make sure you present your evidence well. Use appropriate visualisations, reporting techniques and statistical analysis, as appropriate.

If you visualise, follow the basic rules, as illustrated in Figure \ref{fig:boxplot}:
\begin{itemize}
\item Label everything correctly (axis, title, units).
\item Caption thoroughly.
\item Reference in text.
\item \textbf{Include appropriate display of uncertainty (e.g. error bars, Box plot)}
\item Minimize clutter.
\end{itemize}

See the file \texttt{guide\_to\_visualising.pdf} for further information and guidance.

\begin{figure}
    \centering
    \includegraphics[width=1.0\linewidth]{images/boxplot_finger_distance.pdf}    

    \caption{Average number of fingers detected by the touch sensor at different heights above the surface, averaged over all gestures. Dashed lines indicate
    the true number of fingers present. The Box plots include bootstrapped uncertainty notches for the median. It is clear that the device is biased toward 
    undercounting fingers, particularly at higher $z$ distances.
    }

    % use the notation fig:name to cross reference a figure
    \label{fig:boxplot} 
\end{figure}


%==================================================================================================================================
\chapter{Conclusion}    
Summarise the whole project for a lazy reader who didn't read the rest (e.g. a prize-awarding committee).
\section{Guidance}
\begin{itemize}
    \item
        Summarise briefly and fairly.
    \item
        You should be addressing the general problem you introduced in the
        Introduction.        
    \item
        Include summary of concrete results (``the new compiler ran 2x
        faster'')
    \item
        Indicate what future work could be done, but remember: \textbf{you
        won't get credit for things you haven't done}.
\end{itemize}

%==================================================================================================================================
%
% 
%==================================================================================================================================
%  APPENDICES  

\begin{appendices}

\chapter{Appendices}

Typical inclusions in the appendices are:

\begin{itemize}
\item
  Copies of ethics approvals (required if obtained)
\item
  Copies of questionnaires etc. used to gather data from subjects.
\item
  Extensive tables or figures that are too bulky to fit in the main body of
  the report, particularly ones that are repetitive and summarised in the body.

\item Outline of the source code (e.g. directory structure), or other architecture documentation like class diagrams.

\item User manuals, and any guides to starting/running the software.

\end{itemize}

\textbf{Don't include your source code in the appendices}. It will be
submitted separately.

\end{appendices}

%==================================================================================================================================
%   BIBLIOGRAPHY   

% The bibliography style is abbrvnat
% The bibliography always appears last, after the appendices.

\bibliographystyle{abbrvnat}

\bibliography{l4proj}

\end{document}
