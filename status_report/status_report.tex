    
\documentclass[11pt]{article}
\usepackage{times}
    \usepackage{fullpage}
    
    \title{ A Seamful Game based on GPS Shadows }
    \author{ Robbert Malcolm Sinclair - 2432710S }

    \begin{document}
    \maketitle
    
    
     

\section{Status report}

\subsection{Proposal}\label{proposal}

\subsubsection{Motivation}\label{motivation}

%\emph{{[}Clearly motivate the purpose of your project; why someone would
%care about what you are doing{]}}

The Global Positioning System or GPS has had a huge impact on our daily lives. It helps us navigate around unfamiliar areas with ease. While GPS is accurate most of the time, there
are some areas like in buildings or under bridges where the accuracy can decrease. This is what we refer to as a GPS Shadow. Seamful design is the idea of making what might be
a problem or frustrating as a benefit to the product. 


\subsubsection{Aims}\label{aims}

\emph{{[}Clearly state what the project is intended to do. This should
be something which is measurable; it should be possible to tell if you
succeeded{]}}

The project brief is to create a seamful game utilising areas where GPS Shadows are prevelant. Players will be able to use drops in GPS quality to hide in several different areas
from players wishing to catch them. One of the aims of this game is to show where their signal from a GPS device will drop in quality, for example in buldings or under bridges etc.
Once you know where there are GPS Shadows it should hopefully inform them of a good strategy to use when looking for hiding places.

\subsection{Progress}\label{progress}

\begin{itemize}
    \tightlist
\item Decided on a Technology Stack. Android Phone app on the frontend with a Node.js server and MongoDB database. 
\item Created a Logging app which created a dataset of GPS Shadows around the Glasgow University Campus
\item Compared two different Geolocation APIs and made a decision on the LocationManager API in the android library
\item Came up with some small designs of game ideas
\item Started to make a minimum viable product for the game
\item Deployed a Web Server on Railway.app

\end{itemize}

\subsection{Problems and risks}\label{problems-and-risks}

\subsubsection{Problems}\label{problems}

One of the big problems so far was finding out how to get pure GPS locations. For example the first API that I used, the FusedLocationClient
API, was not suitable because it sourced it's locations from both WiFi and GPS. It was only until I found out about NMEA strings, that I managed
to utilise the LocationManager API to get full GPS spots.
\par
Another problem was that I didn't have as many GPS shadows when I made the switch to the LocationManager API. When I had used the FusedLocationClient
I would have a lot of variation in the amount of shadows. It was especially apparent when walking around Kelvingrove Park. Therefore I had to alter
my design of what kind of places you can hide in to places with a lot of public buildings
\par
During my initial deployment of the project. I used a Django Application on a PythonAnywhere server. However since PythonAnywhere did not support
WebSockets which would have been crucial when developing a game. I then moved the Django server onto a server on the railway.app platform. However
since Django needs a special type of server to run I decided to switch to Node.js so that I can get WebSockets to work and to get access to MongoDB
which allows for easy GeoSpatial Queries.

\subsubsection{Risks}\label{risks}


\begin{itemize}
    \tightlist
    \item I don't have a safeguard if a phone loses internet connection. \textbf{Mitigation:} I will do some research into techniques that will keep player data if it crashes
    \item I'm unsure of the ethical requirements of my evaluation \textbf{Mitigation:} I will read through the checklist when planning my evaluation
    \item I need to be aware about the computational constraints of a smartphone \textbf{Mitigation:} I will have a look at what computations can be done on my web server rather than on my phone to help lighten the load.
\end{itemize}

\subsection{Plan}\label{plan}

\begin{itemize}
    \tightlist
    \item \textbf{Christmas Holidays} Finish a bare minimum minimum viable product tag game. Prepare Evaluation for the game
    \item \textbf{Weeks 1-2} Aim to have a first evaluation on the minimum game around this time
    \item \textbf{Weeks 3-6} Code up improvements to the game as discussed in the evaluation
    \item \textbf{Weeks 6-7} Evaluation of the final game
    \item \textbf{Weeks 8-11} Write Up and final submission of the project.
\end{itemize}


    
\subsection{Ethics and data}\label{ethics}
\emph{Specify what ethical approval you need to do your evaluation and how you are approaching it. This is mandatory. 
Specify what data you expect to collect in your evaluation. Explain how this data will help you evaluate your project.
}

Options for ethics:
\item This project does not involve human subjects or data. No approval required.
\item I have verified that the ethics checklist will apply to any evaluation I need to do. I will sign and complete the checklist.
\item I have sought ethical guidance from the School's ethics convener and I will:
\begin{itemize}
    \item Proceed under specific instructions from the Ethics convener (e.g. modified checklist).
    \item Apply for College Ethics Board approval.
    \item Other procedure (give details)
\end{itemize}    


\end{document}
